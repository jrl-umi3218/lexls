\documentclass[12pt]{article}
\usepackage{fullpage,amsmath,amsfonts,verbatim,balance}
\usepackage[small,bf]{caption}

\usepackage{mathtools}
\usepackage{subfigure}

\usepackage{graphicx,color,psfrag}
\usepackage{epsfig}
\usepackage{hyperref}
\usepackage{color}

\usepackage{accents}
\newcommand{\ubar}[1]{\underaccent{\bar}{#1}}

\usepackage{bbm}

\input defs.tex

\title{Dual approach to solving a hierarchy} \author{Dimitar Dimitrov}

% \date{} % to remove the date

\begin{document}
\maketitle

\section{The dual approach}

For $k=1,\dots,P$ compute a pair $(x_k^{\star},\lambda_{k}^{\star})$ using
%
\begin{align}
\lambda_k^{\star} \in \argmin{\lambda_{k}} & \norm{B_k\lambda_{k} - b_k}^2, \quad
x_k^{\star} = -\frac{1}{\mu_k^2}\sum_{i=1}^{k}A_i^T\lambda_{ki}^{\star},
\end{align}
%
where
%
\begin{align*}
B_k = \begin{bmatrix}A_1^T & \dots & A_k^T \\ 0 & \dots & \mu_kI \end{bmatrix}, \quad
b_k = -\mu_k\begin{bmatrix} \mu_k x_{k-1}^{\star} \\ y_k - A_kx_{k-1}^{\star} \end{bmatrix}, \quad
\lambda_{k} = \begin{bmatrix} \lambda_{k1} \\ \vdots \\ \lambda_{kk} \end{bmatrix} \in \mathbb{R}^{d_k}, \quad
d_k = \sum_{i=1}^{k}m_i.
\end{align*}
%
The recursion in initialized with $x_0^{\star} = 0$. After performing the $P$ steps from above, one
obtains a solution pair $(x^{\star}, \Lambda^{\star})$ with
%
\begin{align}
x^{\star} = x_P^{\star}, \quad
\Lambda^{\star} =
\begin{bmatrix}
  \lambda_{11}^{\star} & \lambda_{21}^{\star} & \dots  &\lambda_{P1}^{\star} \\
                       & \lambda_{22}^{\star} & \dots  &\lambda_{P2}^{\star} \\
                       &                      & \ddots & \vdots \\
                       &                      &        &\lambda_{PP}^{\star} \\
\end{bmatrix}.
\end{align}

\section{A note on our current approach}

We don't compute all the $x_k^{\star}$'s (even though we could) but directly $x^{\star}$. In this
case we can use the condition
%
\[
x^{\star} = -\frac{1}{\mu_k^2}\sum_{i=1}^{k}A_i^T\lambda_{ki}^{\star}
\]
%
for computing $\lambda_{k}^{\star}$.

\newpage

\section{Derivation (of the dual approach)}

\subsection{First objective}

\begin{itemize}

\item The primal problem
  %
  \begin{align}
    \begin{split}
      \minimize{x, \, r_1} & f_1(r_1) = \frac{1}{2}\norm{r_1}^2 + \frac{\mu_1^2}{2}\norm{x}^2\\
      \st & r_1 = A_1x - y_1.
    \end{split}
  \end{align}

\item The Lagrangian
  %
  \[
  L_1(x,r_1,\lambda_{11}) = \frac{1}{2}r_1^Tr_1 + \frac{\mu_1^2}{2} x^Tx + \lambda_{11}^T(A_1x - y_1 - r_1).
  \]

\item The dual function
%
\[
g_1(\lambda_{11}) = \inf_{x,r_1} \underbrace{\frac{1}{2}r_1^Tr_1 - \lambda_{11}^Tr_1} +
\underbrace{\frac{\mu_1^2}{2}x^Tx + \lambda_{11}^TA_1x} - \lambda_{11}^Ty_1.
\]

Minimizing w.r.t. $x$ and then $r_1$ gives
%
\begin{align} \label{eq.x_r}
x^{\star} = -\frac{1}{\mu_1^2}A_1^T\lambda_{11}, \quad r_1^{\star} = \lambda_{11}.
\end{align}
%
Substituting back leads to
%
\begin{align}
\frac{1}{2}r_1^Tr_1 - \lambda_{11}^Tr_1 &= -\frac{1}{2}\lambda_{11}^T\lambda_{11} \\
\frac{\mu_1^2}{2} x^Tx + \lambda_{11}^TA_1x &= -\frac{1}{2\mu_1^2} \lambda_{11}^T A_1A_1^T \lambda_{11}.
\end{align}
%
Hence,
%
\begin{align}
g_1(\lambda_{11}) = -\frac{1}{2\mu_1^2}\lambda_{11}^T\left(\mu_1^2I+A_1A_1^T\right)\lambda_{11} - \lambda_{11}^Ty_1.
\end{align}
%
Note that due to the regularization, the dual function cannot take infinite values and hence there
is no need to have constraints.

\item The dual problem
%
\begin{align}
  \minimize{\lambda_{11}} -g_1(\lambda_{11}) = \frac{1}{2\mu_1^2}\lambda_{11}^T\left(\mu_1^2 I+A_1A_1^T\right)\lambda_{11} + \lambda_{11}^Ty_1,
\end{align}
%
or equivalently
%
\begin{align}
  \minimize{\lambda_{11}} \lambda_{11}^T\left(\mu_1^2 I+A_1A_1^T\right)\lambda_{11} + 2\mu_1^2\lambda_{11}^Ty_1.
\end{align}
%
This can be formulated as the following least-squares problem
%
\begin{align}
\minimize{\lambda_{11}} & \norm{\begin{bmatrix}A_1^T \\ \mu_1I \end{bmatrix}\lambda_{11} - \begin{bmatrix} 0 \\ \mu_1y_1 \end{bmatrix}}^2.
\end{align}

\end{itemize}

\newpage

\subsection{Second objective}


\begin{itemize}

\item The primal problem
  %
  \begin{align}
    \begin{split}
      \minimize{x, \, r_2} & f_2(r_2) = \frac{1}{2}\norm{r_2}^2 + \frac{\mu_2^2}{2}\norm{x}^2\\
      \st & r_2 = A_2x - y_2 \\
      & r_1^{\star} = A_1x - y_1.
    \end{split}
  \end{align}

\item The Lagrangian
  %
  \[
  L_1(x,r_2,\lambda_{21},\lambda_{22}) = \frac{1}{2}r_2^Tr_2 + \frac{\mu_2^2}{2} x^Tx + \lambda_{21}^T(A_1x - y_1 - r_1^{\star}) + \lambda_{22}^T(A_2x - y_2 - r_2).
  \]

\item The dual function
%
\[
g_1(\lambda_{21}, \lambda_{22}) = \inf_{x,r_2} \underbrace{\frac{1}{2}r_2^Tr_2 - \lambda_{22}^Tr_2} + \underbrace{\frac{\mu_2^2}{2}x^Tx + \lambda_{21}^TA_1x + \lambda_{22}^TA_2x} -
\begin{bmatrix} \lambda_{21} \\ \lambda_{22} \end{bmatrix}^T \begin{bmatrix} y_1+r_1^{\star} \\ y_2 \end{bmatrix}.
\]

Minimizing w.r.t. $x$ and then $r_2$ gives
%
\begin{align} \label{eq.x_r}
x^{\star} = -\frac{1}{\mu_2^2}\sum_{i=1}^{2}A_i^T\lambda_{2i}, \quad r_2^{\star} = \lambda_{22}.
\end{align}
%
Substituting back leads to
%
\begin{align}
\frac{1}{2}r_2^Tr_2 - \lambda_{22}^Tr_2 &= -\frac{1}{2}\lambda_{22}^T\lambda_{22} \\
\frac{\mu_2^2}{2} x^Tx + \lambda_{21}^TA_1x + \lambda_{22}^TA_2x
&= -\frac{1}{2\mu_2^2} \begin{bmatrix} \lambda_{21} \\ \lambda_{22} \end{bmatrix}^T
\begin{bmatrix} A_1A_1^T & A_1A_2^T \\ A_2A_1^T & A_2A_2^T \end{bmatrix}
\begin{bmatrix} \lambda_{21} \\ \lambda_{22} \end{bmatrix}.
\end{align}
%
Hence,
%
\begin{align}
g_2(\lambda_{21},\lambda_{22}) =
-\frac{1}{2\mu_2^2} \begin{bmatrix} \lambda_{21} \\ \lambda_{22} \end{bmatrix}^T
\begin{bmatrix} A_1A_1^T & A_1A_2^T \\ A_2A_1^T & \mu_2^2 I+A_2A_2^T \end{bmatrix}
\begin{bmatrix} \lambda_{21} \\ \lambda_{22} \end{bmatrix} -
\begin{bmatrix} \lambda_{21} \\ \lambda_{22} \end{bmatrix}^T \begin{bmatrix} y_1+r_1^{\star} \\ y_2 \end{bmatrix}.
\end{align}

\item The dual problem
%
\begin{align}
\minimize{\lambda_{2}} &
\begin{bmatrix} \lambda_{21} \\ \lambda_{22} \end{bmatrix}^T
\begin{bmatrix} A_1A_1^T & A_1A_2^T \\ A_2A_1^T & \mu_2 I+A_2A_2^T \end{bmatrix}
\begin{bmatrix} \lambda_{21} \\ \lambda_{22} \end{bmatrix} +
2\mu_2^2
\begin{bmatrix} \lambda_{21} \\ \lambda_{22} \end{bmatrix}^T \begin{bmatrix} y_1+r_1^{\star} \\ y_2 \end{bmatrix}.
\end{align}
%
This can be formulated as the following least-squares problem
%
\begin{align}
  \minimize{\lambda_{21},\ \lambda_{22}} & \norm{\begin{bmatrix}A_1^T & A_2^T \\ 0 & \mu_2I \end{bmatrix}
    \begin{bmatrix} \lambda_{21} \\ \lambda_{22} \end{bmatrix} -
    \mu_2\begin{bmatrix} -\mu_2 x_1^{\star} \\ -\left(y_2 - A_2x_1^{\star}\right) \end{bmatrix}}^2.
\end{align}

\end{itemize}

\end{document}

% ---------------------------------------------------------------------------
